\chapter{Conclusion}
%
In a closing statement, the experiment is applicable to investigate the properties of a \textsc{Geiger-Mueller} tube and
clearly shows effects of radioactive decay. The oscilloscope shows the GMTs signal with its characteristic patterns
values as well as the recorded curves beeing very similar to the expected ones.\par
Unfortunately, the maximum adjustable voltage was only \( 600 V \). Therefore, the measurement ended somewhere in the
plateau area. The angular dependency of the count rate is plausible, as fewer particles arrive at an effective spatial
angle. The different absorption materials show that different material properties have a rather huge impact the rays passing
through. While some could drop the count rate to a degree comparable to the blank measurement discribed in \cref{sec:backgroundRadiation}
other seemingly had no impact at all. The raw measurement illustrates the statistical decay processes as well as the
functionality of the GMT.\par
Even though it was already common sense, measuring and visualizing the omnipresent background radiation as well as making
radioactivity of digestibles visible was a fascinating excercise. Plans are rising to give building a cloud chamber another
go.\par\medskip
While we assume it to be rather simple, yet we missed an introduction to the code used on the \micro-controller.