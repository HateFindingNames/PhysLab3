\chapter{Execution}
    \section{Boost Converter}\label{sec:exec:boost_conv}
        To examine the characteristics of the APG, the potentiometer on the circuit board is first set completely
        counterclockwise. The power supply is turned on, so that the duty cycle (in \%) and the output voltage can be taken from
        the LCD. The two values are noted and the potentiometer is turned up until the value for duty cycle has increased by 10 \%.
        Again the values are noted. This process is repeated until the potentiometer is turned completely clockwise.
        The record can be seen in \cref{subtab:3-1_duty_vs_voltage}.
        %
    \section{Avalanche Pulse Generator}\label{sec:exec:avalanche_p_gen}
        The potentiometer is set back to the fully counterclockwise position. Now the oscilloscope is needed and therefore
        switched on. The output of the APG gets connected with the Oscilloscope via a short coaxial cable. The
        potentiometer is slowly turned up until a pulse appears on the oscilloscope. The display is adjusted so that the signal
        can be read easily. A photograph is taken for documentation purposes (\cref{subfig:osci:avalanche_pulse_signal})
        and the voltage shown on the LCD is noted (bottom of \cref{subtab:3-1_duty_vs_voltage}). After that, the voltage
        is set to \(U = \SI{75}{V}\) and a second photograph is taken from the screen (\cref{subfig:osci:pulsuntersuchung}).
        %
    \section{Signal Propagation}\label{sec:exec:signal_prop}
        \subsection{Propagation Time}\label{subsec:exec:prop_time}
            For this experiment a T-piece is inserted between the oscilloscope and the APG. Therefore, the T-piece is
            connected to channel 1 of the oscilloscope and the short coaxial cable from the APG is plugged into the
            T-piece. A second T-piece is connected to channel 2 of the oscilloscope. One end of the T-piece in channel 2 is terminated
            with the \SI{50}{\ohm} terminal resistor, mentioned in the set-up chapter. There is one open end left on each T-piece.
            The three different coaxial cables get connected one after another to these ends. For each cable, there are two pulses
            shown on the oscilloscope. With the cursor function of the oscilloscope, the time delay between the pulses are measured
            and noted (\cref{subtab:3-3-1_propagationTimes_3_cables}). The oscillogram will look like \cref{subfig:osci:ch1-ch2_delayed_pulse}.
            %
        \subsection{Cable Characteristics}\label{subsec:exec:cable_char}
            To investigate the cable characteristics, the length of the three cables given is measured with the tape measure.
            Afterwards they are connected one after another with the T-piece at channel 1. Channel 2 is not needed during this
            measurement. For each cable two photographs are taken from the oscilloscopes screen. The first photograph with an open
            end (as in \cref{subfig:osci:3-3-2_open_reflectionTime}), the second with a short end (as in \cref{subfig:osci:3-3-2_shorted}).
            To short-circuit the end of the coaxial cable, a screwdriver is used. With the cursor function, the propagation
            time \(\tau_0\) (for open end) and \(\tau_s\) (for short-circuited end) are read from the screen and noted.\par
            Now, the termination box is connected to the open end of the cable. The potentiometer on the termination box is rotated
            while looking at the oscilloscopes screen. Once the oscilloscope shows a minimum amplitude of the reflected pulse (as in \cref{subfig:osci:3-3-2_Z_0_equal_R_term}), the
            setting of the potentiometer is not changed anymore. The termination box is removed and then connected to the multimeter.
            The multimeter is set so that the resistance of the termination box can be read from it. This procedure is repeated for
            the other remaining cables.
            %
        \subsection{Time Domain Reflectometry}\label{subsec:exec:TDR}
            The cable with unknown length and an internal fault is now connected to the T-piece at channel 1. The first cursor of the
            oscilloscope is set to the origin pulse. The second cursor is first set to the pulse of the reflection of the internal
            fault (\cref{subfig:osci:3-3-3_lengthToDefect}). The time delay is noted. Then the second cursor is set to the pulse
            of the reflection of the cables end (\cref{subfig:osci:3-3-3_overallLength}). The time delay is noted again.