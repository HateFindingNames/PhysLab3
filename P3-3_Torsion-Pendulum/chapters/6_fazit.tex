\chapter{Conclusion}
%
In retrospect, the purpose of the experiment can be considered cautiously as achieved. Although there were some problems,
the properties of a torsional pendulum were successfully investigated.
Initially, the characteristic curve recording of the sensor worked well, thus confirming the functionality of the capacitor.
Unfortunately, from the determination of the damped vibration, a few problems occurred with the program \textsc{RealTerm} and the
\micro C. The  timer  ticks no longer corresponded to the angle. Nevertheless, the curve could be recorded cleanly
and the period time could be determined normally, because the amplitude did not play a role in the evaluation. As expected,
the period times had only small deviations for the different currents. The damping coefficients were also plausible, as
the damping coefficient increased exponentially as the current increased.\par
Furthermore, the dimensioning of the inertia of the three rods also worked. But the aluminum rod seemed to have too little
inertia, as its period time was similar to that of the pendulum without a rod. As a result, no pendulum inertia could be
determined depending on the aluminum rod. The two inertia obtained differ from each other and the error range of the two
do not cover either. This could also be due to the inaccurate reading of the period time. Nevertheless, the order of
magnitude seems to be correct.\par
Another unnoticed cause of the error could be the wear and aging of the setup, as the wire, for example, looked quite
sensitive and worn out.\par
It can then be clearly stated that the \textsc{RealTerm} program should be revised and that several, heavier and longer rods should
be used to better determine the unknown pendulum inertia.\par\medskip
The estimated sensitivity is off by quite a bit. It is assumed, that the resistor does not play a significant role estimating
the sensitivity.\par
Nevertheless, taking the sensitivity obtained from the data and plugged into \cref{eq:est.sensitivity} yield a value of
\(R \approx \SI{10^7}{\ohm}\). Considering the very small capacitances, it appears reasonable to have a high resistance
in order not to charge the capacitor too quick.